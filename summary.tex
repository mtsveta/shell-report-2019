%% LyX 2.3.3 created this file.  For more info, see http://www.lyx.org/.
%% Do not edit unless you really know what you are doing.
\documentclass[12pt,english]{article}
\usepackage{garamondx}
\usepackage{tgcursor}
\usepackage[garamondx]{newtxmath}
\usepackage[T1]{fontenc}
\usepackage[latin9]{inputenc}
\usepackage{geometry}
\geometry{verbose,tmargin=2cm,bmargin=2cm,lmargin=2cm,rmargin=2cm}
\setlength{\parskip}{\smallskipamount}
\setlength{\parindent}{0pt}
\synctex=-1
\usepackage{babel}
\usepackage[authoryear]{natbib}
\usepackage[unicode=true,
 bookmarks=true,bookmarksnumbered=false,bookmarksopen=false,
 breaklinks=false,pdfborder={0 0 0},pdfborderstyle={},backref=false,colorlinks=false]
 {hyperref}
\hypersetup{
 pdfauthor={Allan M. M. Leal}}

\makeatletter
\@ifundefined{date}{}{\date{}}
\makeatother

\begin{document}

\section*{Summary}

The main goals of this project are:
\begin{itemize}
\item Accelerate geochemical reaction calculations (chemical equilibrium
and chemical kinetics) in reactive transport simulations using a novel
on-demand machine learning algorithm.
\item Implement missing geochemical capabilities and models in Reaktoro
such as redox reactions, exchange, surface complexation, solid solutions,
and other modeling capabilities. 
\item Implement native support for PHREEQC databases in Reaktoro.
\end{itemize}
In this past year (the first year in this three-year project), we
have primarily focused on advancing the first goal above (the most
challenging deliverable among the others). We summarize below the
results so far achieved, followed by the major tasks we will concentrate
next year. 
\begin{center}
\textbf{\emph{\small{}The summary below aims to be self contained
in an effort to enable those not so involved in this project to understand
our latest achievements in a quicker way. Some parts, however, may
require further reading in the subsequent sections for proper understanding.}}{\small\par}
\par\end{center}

\subsection*{Achievements in the past year}

\subsubsection*{\emph{Further improvements in the accelerated chemical equilibrium
algorithm }}

Accelerating chemical equilibrium calculations will always be mission
critical in the simulation of reactive transport phenomena. This is
because assuming aqueous and gaseous species (and also some fast reacting
minerals) to react under an instantaneous chemical equilibrium model
is a suitable simplification. The reason for this is that the rates
of those reactions are in general much faster than the rates of subsurface
transport processes. 

We extensively tested the on-demand machine learning algorithm (ODML)
in a 1D reactive transport problem, where all chemical reaction processes
were assumed to be controlled by chemical equilibrium. We considered
a mesh with 100 cells, and executed 10,000 time steps. This required
1,000,000 equilibrium calculations, one per mesh cell and per time
step. A Pitzer activity model was used to represent the non-ideal
thermodynamic behavior of the aqueous solution. The choice for using
Pitzer model is because of its superior accuracy compared to others
(Davies, Debye-H�ckel), but also because it is one of most computationally
expensive aqueous activity models in the literature. The chemical
system consisted of 36 species in 4 phases (an aqueous phase and three
mineral phases, namely, calcite, dolomite and quartz). We obtained
the following results: 
\begin{itemize}
\item a speedup of \textbf{200}\textbf{\emph{x}} (in the worst case) and
\textbf{400}\textbf{\emph{x}} (in the best\slash zero-cost-search
case) for the chemical equilibrium calculations in the reactive transport
simulations. 
\item only \textbf{258} chemical equilibrium states out of \textbf{1,000,000}
were fully evaluated (by triggering a Gibbs energy minimization computation);
the rest, \textbf{99.97\%}, were quickly and accurately predicted. 
\item the relative error of the predictions is less than 1\% for mineral
phase volumes and less than 4\% for aqueous species concentrations. 
\end{itemize}

\subsubsection*{\emph{Manuscript submitted for publication}}

A manuscript entitled \emph{Accelerating Reactive Transport Modeling:
On-Demand Machine Learning Algorithm for Chemical Equilibrium Calculations
}was submitted to the journal of Transport in Porous Media on November
22nd, 2019. This manuscript presents our latest achievements on accelerated
reactive transport simulations when using ODML to speed up exclusively
chemical equilibrium calculations. A summary of this manuscript is
presented in this report. We plan to submit a follow up manuscript
next year on accelerating chemical kinetics in reactive transport
simulations.

\subsubsection*{\emph{Chemical kinetics accelerated with ODML }}

We have started the work on using ODML algorithm to speed up chemical
kinetics of mineral dissolution\slash precipitation reactions in reactive
transport simulations. Our preliminary numerical experiments, using
a similar reactive transport problem, with calcite dissolving kinetically,
produced the following results: 
\begin{itemize}
\item a speedup of \textbf{8-10}\textbf{\emph{x}} for the chemical kinetics
computations (compared to the conventional algorithm implemented in
Reaktoro); 
\item a speedup of \textbf{30}\textbf{\emph{x}} when combining accelerated
chemical kinetics of mineral dissolution with accelerated chemical
equilibrium calculations for the aqueous species.
\end{itemize}
We also have preliminary results on the use of ODML when more than
one kinetically-controlled mineral reaction is considered:
\begin{itemize}
\item a speed up of \textbf{35-102}\textbf{\emph{x}} with two kinetically-controlled
minerals; and
\item a speed up of \textbf{92-124}\textbf{\emph{x}} with tree kinetically-controlled
minerals. 
\end{itemize}
\textbf{Remark: }These results are a product of a few months of numerical
experimentation, and the speedups listed above can be increased with
our research plans and ideas outlined later.

\subsubsection*{\emph{Rok, a reactive transport simulator (in its early stages) using
Reaktoro and Firedrake}}

Together with a collaborator, an expert in finite element methods
and a long time user and contributor of Firedrake (\texttt{\href{http://firedrakeproject.org}{firedrakeproject.org}}),
a Python library for solving partial differential equations, we developed
a proof of concept of a reactive transport simulator: \textbf{Rok}.
In this newly developed code, Reaktoro is used for the chemical reaction
calculations and Firedrake for the solution of the coupled governing
partial differential equations (species transport, Darcy equation,
etc.). This simulator has allowed us to perform more interesting reactive
transport simulations, considering higher-dimensions (2D) and strongly
heterogeneous porous media. We plan to use Rok not only for simulations
in general, but also as a testing platform to ensure that the ODML
algorithms we are developing work optimally under more challenging
geochemical and geologic conditions. 

We believe this project will also be helpful to Shell, as it will
demonstrate how Reaktoro can be coupled with other codes to develop
reactive transport simulators. 

\subsection*{Planned research tasks for next year}

\subsubsection*{\emph{Improving the search operations in the algorithm}}

In the on-demand learning algorithm, a new chemical equilibrium\slash kinetics
calculation is performed by first searching for previously solved
problems with similar input conditions. We currently use a nearest
neighbor search approach, in which Euclidean distances are used to
compare the new input vector with input vectors previously computed.
The nearest neighbor problem is then used to estimate the solution
of the new problem, using exact sensitivity analysis. 

We have observed, however, that sometimes the nearest neighbor is
not necessarily the most adequate reference point from which a prediction
is performed, which end up causing many unnecessary training operations.
This is particularly more pronounced when the complexity of the chemical
system increases, which leads to an increased dimensionality of the
input space. Our research plan to resolve this issue is to explore
alternative search procedures, such as, for example, rank each previously
solved problem in terms of how often it is used as a reference problem
to successfully predict others. We could start our search by testing
these key problems first. We could even limit the maximum number of
stored trained problems. Only those that have been been successfully
used to estimate\slash predict others and has a \emph{rate-of-use}
higher than a tolerance would be kept in a dynamic database. 

\subsubsection*{\emph{Improving the acceptance criteria in the algorithm}}

The on-demand learning algorithm is able to entirely bypass the iterative
computing processes of solving chemical equilibrium and kinetics problems.
The approximated solutions of these problems need, however, to be
accepted or rejected based on some given \emph{acceptance criteria}.
A rejection triggers a new training operation, which is basically
the solution of the problem in a conventional way (e.g. using a Gibbs
energy minimization algorithm), followed by a sensitivity analysis
operation. 

Currently, our acceptance criteria sometimes trigger more training
operations than it should (i.e. some predictions could have been safely
accepted, but are otherwise marked as not accurate enough). We plan
to improve the acceptance criteria in the ODML algorithm by using,
for example, \emph{measures of disequilibrium of reactions}. For example,
if the saturation index of a mineral that should be in equilibrium
is estimated to be sufficiently far from equilibrium, we then decide
that the estimated equilibrium state is not accurate enough, and it
must be exactly computed, using a Gibbs energy minimization (GEM)
algorithm. 

This proposed test is potentially more adequate than the one we currently
use: if predicted species activities and reference activities change
by more than, say, 10\%, we declare the ODML prediction is not accurate
enough and a new training is performed. 

\subsubsection*{\emph{Accelerate chemical equilibrium calculations when evaluating
reaction rates}}

Reaction rates of mineral dissolution\slash precipitation processes
require, in general, the saturation index of the minerals. This in
turn require chemical equilibrium calculation for the aqueous\slash gaseous
species. We plan to accelerate the evaluation of reaction rates by
speeding up these chemical equilibrium calculations, which may be
needed many times during a single kinetic time step. 

\subsubsection*{\emph{Test the on-demand learning algorithm on more complex scenarios}}

So far, we have mainly used a 1D reactive transport example, representative
of a dolomitization process, to test the accelerated chemical equilibrium\slash kinetics
algorithms. We plan to use Rok for testing the accelerated chemical
algorithms in higher dimensions, in more complex geologic conditions.
We also plan to model a highly complex reactive transport process
resulting from the the attack of seawater into cement and concrete
systems (with collaborators from the Paul Scherrer Institute, Switzerland).
In this problem, we will consider \textasciitilde 100 aqueous species,
\textasciitilde 60 pure minerals, and \textasciitilde 10 solid solutions. 

Furthermore, we will also consider more complex cases for chemical
kinetics, including more minerals under kinetic control. 

\subsubsection*{\emph{Other numerical and computational tasks}}

We also plan to:
\begin{itemize}
\item further develop Rok to permit 3D reactive transport simulations with
heat transfer capabilities;
\item improve support for modeling redox reactions in Reaktoro;
\item decoupling of specific surface area from the reaction definition,
so that different surface areas (for each mineral) can be set across
the reservoir if needed;
\item integrating our newly developed numerical optimization library, Optima,
into Reaktoro for more robust and faster Gibbs energy minimization
calculations;
\item compare the backward differentiation formula (BDF) schemes provided
by CVODE (the solver we currently use to solve ordinary differential
equations) with RADAU algorithms (higher-order, implicit in time,
Runge-Kutta methods) for numerical integration of the system of ODEs,
in order to select the best solver for stiff ODE-systems describing
kinetically controlled species;
\item compare the performance of Reaktoro and PHREEQC on a set of benchmarks
provided by Shell;
\item together with collaborators from the Paul Scherrer Institute, improve
the support of thermodynamic databases in Reaktoro, including those
from PHREEQC.
\end{itemize}
\begin{center}
\textbf{\emph{Priority will be given to tasks related to extending
the modeling capabilities of Reaktoro and those that further improve
the on-demand learning algorithms for faster chemical equilibrium\slash kinetics
calculations. }}
\par\end{center}
\end{document}
